\documentclass{article}
\usepackage[utf8]{inputenc}
\usepackage[margin=1in]{geometry}
\usepackage{amsmath, amssymb, amsthm, graphicx, float}
\usepackage{subfig, subfloat}
\allowdisplaybreaks

\title{Monte-Carlo Based MGS-MR Detection}
\author{Krishna Chaitanya(kg453) \\ Arjun Jauhari(aj526)}
\date{4 December 2015}

\begin{document}

\maketitle

\section*{Abstract}
In this project we implemented and evaluated the algorithm 
proposed in paper "A Novel Monte-Carlo-Sampling-Based Receiver 
for Large-Scale Uplink Multiuser MIMO Systems" by Tanumay Datta et.al. 
We have implemented the algorithm proposed and compared our simulation results
with the one claimed in paper. We were able to reproduce the results published in
paper to a fair extent and therefore agree with authors claim of providing near ML performance
with lower complexity than Sphere decoder in Large Scale Uplink MIMO. \\

\section*{Introduction}
This paper proposes a low-complexity algorithms based on Markov Chain
Monte Carlo(MCMC) sampling for signal detection
on the uplink in large-scale multiuser
multiple-input–multiple-output (MIMO) systems with tens to hun-
dreds of antennas at the base station (BS) and a similar number of
uplink users. A novel mixed sampling
technique (which makes a probabilistic choice between Gibbs sam-
pling and random uniform sampling in each coordinate update)
for detection is proposed. The algorithm proposed for detection
alleviates the stalling problem encountered at high signal-to-noise
ratios (SNRs) in conventional Gibbs-sampling-based detection and
achieves near-optimal performance in large systems with M -ary
quadrature amplitude modulation (M -QAM). A novel ingredi-
ent in the detection algorithm that is responsible for achieving
near-optimal performance at low complexity is the joint use of
a mixed Gibbs sampling (MGS) strategy coupled with a multiple
restart (MR) strategy with an efficient restart criterion. \\

\section*{Background}
Markov Chain Monte-Carlo(MCMC) techniques is a way to sample from a probability
distribution. The probability distribution to be sampled is modeled as the stationary 
distribution of underlying Markov Chain. Gibbs Sampling is one way most popular way
of implementing MCMC. \\

\begin{figure}[H]
\centering
\includegraphics[width=6cm]{w1.png}
\caption{Gibbs Sampler Updates}
\label{fig1:Gibbs Sampler}
\end{figure}

Conventional Gibbs Sampling starts with random initial vector and element wise 
update is done. Updates are done by sampling the distribution as shown in Figure 1.
Symbol vector is updated iteratively and it is guaranteed to converge but can take
infinite many iterations. In practical scenarios we generally stop after a fixed
number of iterations and hope that chain has mixed(converge). However, there can
be cases where chain gets trapped in some low transition probability state and
takes many iteration to come out. Such cases cause Stalling Problem which degrades
Conventional Gibbs sampler performance at high SNR. Figure 2 below shows this problem.

\begin{figure}[H]
\centering
\includegraphics[width=10cm]{w2.png}
\caption{Stalling Problem}
\label{fig2:Stalling Problem}
\end{figure}

\section*{Algorithm}
\subsection*{Mixed Gibbs Sampling}
At each co-ordinate update, instead of updating $x_i$ with probability 1 
as done in conventional Gibbs sampling, update with probability $1-q$ and
use different update rule with probability $q$. The mixed distribution is -
\begin{equation*}
\begin{split}
    &p(x_1,...,x_{2K}|\mathbf{y},\mathbf{H}) \propto (1-q)\psi(\alpha_1) + q\psi(\alpha_2)\\
    &\psi(\alpha) = exp({ -||\mathbf{y - Hx}|| }^2/\alpha^2\sigma^2)\\
\end{split}
\end{equation*}
$\alpha_1$ and $\alpha_2$ are chosen as 1 and $\infty$ respectively.

\begin{figure}[h]
\centering
\subfloat[MGS\label{fig3:MGS}]{%
\includegraphics[width=8cm]{w3.png}
}~
\subfloat[MGS-MR\label{fig3:mgs_mr}]{%
\includegraphics[width=7cm]{w4.png}
}%
\caption{}
\end{figure}

Figure 3a shows how Mixed Gibbs Sampling can reach ML cost in many intermediate
iterations but conventional Gibbs Sampling remain stalled at some high loss state.
Therefore MGS can achieve near optimal performance. \\
But BER degrades as the QAM size increases and MGS is far from optimal performance 
for 16-QAM and 64-QAM. This happens because search space increases exponentially with
constellation size and therefore probability to converge to right solution is low.

\subsection*{Mixed Gibbs Sampling-Multiple Restart}
To get around the above mentioned problem, mutliple restarts are used. Multiple
Restarts is equivalent to running multiple Gibbs Sampler parallely with different
initial vector. Figure 3b shows that if we run multiple gibbs sampler with different
initial vector the probability that one of them will converge is fairly high.\\
Figure 4 compares the performance of MGS and MGS-MR for 16-QAM. As can be seen that
MGS-MR alleviates the problem faced by MGS for higher QAM.\\
\begin{figure}[h]
\centering
\subfloat[MGS vs MGS-MR for 16-QAM\label{fig4:comparison}]{%
\includegraphics[width=8cm]{r3.png}
}~
\subfloat[MGS-MR\label{fig4:complex}]{%
\includegraphics[width=8cm]{c3.png}
}%
\caption{}
\end{figure}

\section*{Results}
\begin{figure}[H]
\centering
\includegraphics[width=12cm]{r4.png}
\caption{MT=8,MR=8 - QPSK}
\label{fig5:Stalling Problem}
\end{figure}
\begin{figure}[H]
\centering
\includegraphics[width=14cm]{r5.png}
\caption{MT=8,MR=16 - QPSK}
\label{fig6:Stalling Problem}
\end{figure}
\begin{figure}[H]
\centering
\includegraphics[width=14cm]{r6.png}
\caption{MT=32,MR=32 - QPSK}
\label{fig7:Stalling Problem}
\end{figure}

\section*{Conclusion}
We implemented the algorithm proposed successfully and demonstrated its performance
under various combination of Transmit and Recieve Antennae. We also estimated the
complexity of the algorithm and it is seen that the complexity is of the order $10^7$
We also studied the problem encountered by MGS algorithm for higher QAMs and 
implemented MGS-MR to see if it improves the performance as claimed by author.\\
\section*{References}
[1] Tanumay Datta et al., A Novel Monte-Carlo-Sampling-Based Receiver 
for Large-Scale Uplink Multiuser MIMO Systems \\
\end{document}

